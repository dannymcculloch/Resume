%-------------------------
% Resume in Latex
% Author : Danny
%------------------------

\documentclass[letterpaper,11pt]{article}

\usepackage{latexsym}
\usepackage[empty]{fullpage}
\usepackage{titlesec}
\usepackage{marvosym}
\usepackage[usenames,dvipsnames]{color}
\usepackage{verbatim}
\usepackage{enumitem}
\usepackage[hidelinks]{hyperref}
\usepackage{fancyhdr}
\usepackage[english]{babel}
\usepackage{tabularx}
% only for pdflatex
% \input{glyphtounicode}

% fontawesome
\usepackage{fontawesome5}

% fixed width
\usepackage[scale=0.90,lf]{FiraMono}

% light-grey
\definecolor{light-grey}{gray}{0.83}
\definecolor{dark-grey}{gray}{0.3}
\definecolor{text-grey}{gray}{.08}

\DeclareRobustCommand{\ebseries}{\fontseries{eb}\selectfont}
\DeclareTextFontCommand{\texteb}{\ebseries}

% custom underilne
\usepackage{contour}
\usepackage[normalem]{ulem}
\renewcommand{\ULdepth}{1.8pt}
\contourlength{0.8pt}
\newcommand{\myuline}[1]{%
  \uline{\phantom{#1}}%
  \llap{\contour{white}{#1}}%
}


% custom font: helvetica-style
\usepackage{tgheros}
\renewcommand*\familydefault{\sfdefault} 
%% Only if the base font of the document is to be sans serif
\usepackage[T1]{fontenc}


\pagestyle{fancy}
\fancyhf{} % clear all header and footer fields
\fancyfoot{}
\renewcommand{\headrulewidth}{0pt}
\renewcommand{\footrulewidth}{0pt}

% Adjust margins
\addtolength{\oddsidemargin}{-0.5in}
\addtolength{\evensidemargin}{0in}
\addtolength{\textwidth}{1in}
\addtolength{\topmargin}{-.5in}
\addtolength{\textheight}{1.0in}

\urlstyle{same}

\raggedbottom
\raggedright
\setlength{\tabcolsep}{0in}

% Sections formatting - serif
% \titleformat{\section}{
%   \vspace{2pt} \scshape \raggedright\large % header section
% }{}{0em}{}[\color{black} \titlerule \vspace{-5pt}]

% sans serif sections
\titleformat {\section}{
    \bfseries \vspace{2pt} \raggedright \large % header section
}{}{0em}{}[\color{light-grey} {\titlerule[2pt]} \vspace{-4pt}]

% only for pdflatex
% Ensure that generate pdf is machine readable/ATS parsable
% \pdfgentounicode=1

%-------------------------
% Custom commands
\newcommand{\resumeItem}[1]{
  \item\small{
    {#1 \vspace{-1pt}}
  }
}

\newcommand{\resumeSubheading}[4]{
  \vspace{-1pt}\item
    \begin{tabular*}{\textwidth}[t]{l@{\extracolsep{\fill}}r}
      \textbf{#1} & {\color{dark-grey}\small #2}\vspace{1pt}\\ % top row of resume entry
      \textit{#3} & {\color{dark-grey} \small #4}\\ % second row of resume entry
    \end{tabular*}\vspace{-4pt}
}

\newcommand{\resumeSubSubheading}[2]{
    \item
    \begin{tabular*}{\textwidth}{l@{\extracolsep{\fill}}r}
      \textit{\small#1} & \textit{\small #2} \\
    \end{tabular*}\vspace{-7pt}
}

\newcommand{\resumeProjectHeading}[2]{
    \item
    \begin{tabular*}{\textwidth}{l@{\extracolsep{\fill}}r}
      #1 & {\color{dark-grey}} \\
    \end{tabular*}\vspace{-4pt}
}

\newcommand{\resumeSubItem}[1]{\resumeItem{#1}\vspace{-4pt}}

\renewcommand\labelitemii{$\vcenter{\hbox{\tiny$\bullet$}}$}

% CHANGED default leftmargin  0.15 in
\newcommand{\resumeSubHeadingListStart}{\begin{itemize}[leftmargin=0in, label={}]}
\newcommand{\resumeSubHeadingListEnd}{\end{itemize}}
\newcommand{\resumeItemListStart}{\begin{itemize}}
\newcommand{\resumeItemListEnd}{\end{itemize}\vspace{0pt}}

\color{text-grey}

%-------------------------------------------
%%%%%%  RESUME STARTS HERE  %%%%%%%%%%%%%%%%%%%%%%%%%%%%


\begin{document}

%----------HEADING----------
\begin{center}
    \textbf{\Huge Danny McCulloch} \\ \vspace{5pt}
    \hspace{1pt} \faEnvelope \hspace{2pt} \texttt{dm575@exeter.ac.uk} 
    \hspace{1pt} \faGithub \hspace{2pt} \texttt{dannymcculloch.github.io} 
    \hspace{1pt} $|$ 
    \hspace{1pt} \faMapMarker* \hspace{2pt}\texttt{Exeter, UK}
    \hspace{1pt} $|$ 
    \hspace{1pt} \faTwitter\texttt{TheMartianSub}
    \\
    \texttt{University of Exeter}
    \hspace{1pt}$|$ 
    \texttt{Department of Physics \& Astronomy}
    \\ \vspace{-3pt}
\end{center}

%-----------INTRODUCTION-----------

I am an early-career researcher studying the modern-day Martian climate using the Met Offices Unified Model. I have a strong interest in Martian studies and am personally interested in human exploration on Mars, my long-term ambition is to be a key contributor in making this a reality.

%-----------EDUCATION-----------
\section {EDUCATION}
  \resumeSubHeadingListStart
    \resumeSubheading
      {MSc by Research in Physics}{Sep 2020 -- Sep 2022}
      {\textup{Supervisors: Prof. Nathan Mayne, Prof. Matthew Bate}}{University of Exeter, UK}
      	%\resumeItemListStart
    	\resumeItem {\textbf{Thesis title}: \textit{"Modelling an idealised Martian climate using the Met Office Unified Global Climate Model (UM)"}}
        \resumeItem 
            {\textbf{Research}: Adapting the UM, traditionally used for an Earth climate, for an idealised Mars. This involved:}
            \resumeItemListStart
                \resumeItem {Analysis of high-resolution global climate model (GCM) simulations}
                \resumeItem {Model adaptation and verification using other GCM outputs and satellite data}
                \resumeItem {Adaptation of satellite data for UM compatibility (which is Fortran based)}
                \resumeItem {'Terraforming' using the UM, (i.e. forcing controlled changes to certain primary climate processes and observing their effect on secondary climate processes)}
            \resumeItemListEnd
        %\resumeItemListEnd
  \resumeSubHeadingListEnd
  \resumeSubHeadingListStart
    \resumeSubheading
      {BSc Zoology}{Sep 2017 -- Jul 2020}
      {\textup{2:1 $|$ with Honours}}{University of Exeter, UK}
      	%\resumeItemListStart
    	\resumeItem {\textbf{Dissertation title}: \textit{"Does a decrease in sea ice lead to an increase of phytoplankton in the Southern Ocean?"}}
        \resumeItem 
            {\textbf{Research}:}
            \resumeItemListStart
                \resumeItem {Interpreting and analysing Antarctic sea ice satellite data (mainly AQUA-MODIS)}
                \resumeItem {Interpreting and analysing chlorophyll-a satellite data (multiple satellite sources)}
                \resumeItem {Statistical analysis on temporal and spatial changes to both parameters individually and with their relationship to each other}
                \resumeItem {Ocean circulation and fluid dynamics (led to an  \underline{\hyperref[sec:internships]{internship}})}
                
            \resumeItemListEnd
        %\resumeItemListEnd
  \resumeSubHeadingListEnd

%-----------EXPERIENCE-----------
\section{TEACHING ASSISTANT EXPERIENCE}
  \resumeSubHeadingListStart
    \resumeSubheading
      {"Frontiers in Science"}{Sep 2020 -- Sep 2022}
      {Second year Natural Sciences module}{}
    \resumeSubheading
      {"IT and Electronic skills"}{Sep 2021 -- Sep 2022}
      {First year Physics module}{}
    \resumeSubheading
      {"Scientific programming in Python"}{Sep 2021 -- Sep 2022}
      {Second year Natural Sciences module}{}
    \resumeSubheading
      {"Physics lab demonstrator"}{Sep 2021 -- Sep 2022}
      {First year Physics module}{}
    \resumeSubheading
      {"Mathematics and computing: Integrative Tools for Natural Sciences"}{Sep 2021 -- Sep 2022}
      {First year Natural Sciences module}{}
  \resumeSubHeadingListEnd
\textbf{Tasks involved}: 
\resumeItemListStart 
\resumeItem{Marking reports} 
\resumeItem{Assistance with \textbf{Python} and experiment design} 
\resumeItem{Providing constructive feedback on submitted work} 
\resumeItem{Providing academic support for students}  
\resumeItem{Secondary reviewing of feedback given by demonstrators and academics across the year}
\resumeItemListEnd

%-----------INTERNSHIPS-----------

\section{INTERNSHIPS}
\label{sec:internships}
    \resumeSubHeadingListStart
      \resumeProjectHeading
         {\textbf{Mars GCM characterising -- Prof Nathan Mayne}}{Sep 2019 -- Jun 2020}
          \resumeItemListStart
            \resumeItem{Collated output scenarios from different Mars GCMs, namely; the LMD (Laboratoire de Météorologie Dynamique) General GCM and the NASA AMES models}
            \resumeItem{Characterised the differences between GCMs, mainly focusing on how different GCMs might do the same thing in different ways (e.g. vertical height and gridding)}
            \resumeItem{Internship helped prepare for my eventual Masters course)}

          \resumeItemListEnd
          
        \resumeProjectHeading
          {\textbf{Arctic shipping pollutant monitoring -- Dr Jo Browse}}{Sep 2019 -- Jun 2020}
          \resumeItemListStart
            \resumeItem{Data procurement and analysis on buoy data monitoring sulfates in the Arctic Ocean, study was investigating shipping traffic and the effectiveness of scrubbers}
            \resumeItem{Instrument documentation for the different buoys and the satellite instruments that would be used to conduct/validate the study}
          \resumeItemListEnd
          
      \resumeProjectHeading
          {\textbf{Ocean circulation experiment design -- Dr Katy Sheen}} {Sep 2018 -- Jun 2019}
          \resumeItemListStart
            \resumeItem{Built and turntable and designed experiments to demonstrate fluid dynamic processes which occur commonly in the ocean (e.g. Ekman transport, thermohaline circulation...)}
            \resumeItem{Apparatus and material was to be used for a 3rd year Geography module and for outreach to get people interested in how our oceans work in the local area}
          \resumeItemListEnd
          
    \resumeSubHeadingListEnd



%
%-----------SKILLS-----------
\section{SKILLS}
 \begin{itemize}[leftmargin=0in, label={}]
    \small{\item{
     \textbf{Languages:} {English (native), Spanish (fluent) -- Coding: Python, R, basic Unix}\vspace{2pt} \\
     \textbf{Tools}{: Numerical Global Climate Models, Python; [Iris, SciPy, NumPy, Matplotlib]}
     \\
     \textbf{Personal hobbies}{: Electronics, Snowboarding, Outdoor activities}
    }}
 \end{itemize}
%------------------PUBLICATIONS-----------------
\section{PUBLICATIONS}
 \begin{itemize}[leftmargin=0in, label={}]
    \small{\item{
        McCulloch, D., Mayne, N., Bate, M., and Sergeev, D.: Modelling an idealised Martian climate with the Unified Model: The next “giant leap” for Mars GCMs., European Planetary Science Congress 2021, online, 13–24 Sep 2021, EPSC2021-232, https://doi.org/10.5194/epsc2021-232, 2021. \\
        }}
     \end{itemize}

%------------------REFERENCES-----------------
\section{REFERENCES}
 \begin{itemize}[leftmargin=0in, label={}]
    \small{\item{
     Prof. \textbf{Nathan Mayne}{ $|$ Primary supervisor $|$ Associate Professor in Astrophysics and Planetary Climates:} {N.J.Mayne@exeter.ac.uk $|$ 01392 726244}\vspace{2pt} \\
     Prof. \textbf{Matthew Bate}{ $|$ Secondary supervisor $|$ Head of Astrophysics \& Professor of Theoretical Astrophysics:} {M.R.Bate@exeter.ac.uk $|$ 01392 725513}\vspace{2pt} \\
     Dr. \textbf{Denis Sergeev}{ $|$ Associate $|$ Postdoctoral research fellow:}
     \\
     {D.Sergeev@exeter.ac.uk $|$ 01392 723612}\vspace{2pt} \\

    }}
 \end{itemize}
%-------------------------------------------
\end{document}
